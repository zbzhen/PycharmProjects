\documentclass[10pt,reqno, final]{amsart}
\usepackage[notcite,notref]{showkeys}

\usepackage{url,hyperref,multirow}
\usepackage{color}
\usepackage{stmaryrd}
\usepackage{exscale}

\usepackage{relsize}

\usepackage{epsfig,subfigure,amssymb,amsmath,version,slashbox}
\usepackage{amssymb,version,graphicx,fancybox,mathrsfs,pifont,booktabs}%,wrapfig}

\usepackage{epstopdf}
%\usepackage[active]{srcltx}

%\textwidth 6in
% \textheight 8in
%\renewcommand{\baselinestretch}{1.2}
%\usepackage{graphicx}

\headheight=3pt
\topmargin=0.3cm
\textheight=8.5in
\textwidth=5.8in
\setlength{\oddsidemargin}{1cm}
\setlength{\evensidemargin}{1cm}
\addtolength{\voffset}{-5pt}

\catcode`\@=11 \theoremstyle{plain}
%\@addtoreset{equation}{section}   % Makes \section reset 'equation' counter.
%\renewcommand{\theequation}{\arabic{section}.\arabic{equation}}
%\@addtoreset{figure}{section}
%\renewcommand\thefigure{\thesection.\@arabic\c@figure}
%\@addtoreset{table}{section}
%\renewcommand\thetable{\thesection.\@arabic\c@table}


\renewcommand{\theequation}{\thesection.\arabic{equation}}
\newtheorem{lemma}{Lemma}[section]
\newtheorem{theorem}{Theorem}[section]
\newtheorem{corollary}{Corollary}[section]
\newtheorem{proposition}{Proposition}[section]
\newtheorem{definition}{Definition}[section]
\newtheorem{remark}{Remark}[section]
\newtheorem{example}{Example}[section]
\@addtoreset{figure}{section}
\renewcommand\thefigure{\thesection.\@arabic\c@figure}
\@addtoreset{table}{section}
\renewcommand\thetable{\thesection.\@arabic\c@table}




\newcommand{\bs}[1]{\boldsymbol{#1}}
\def \ri {{\rm i}}



\newcommand{\comm}[1]{\marginpar{%
\vskip-\baselineskip %raise the marginpar a bit
\raggedright\footnotesize
\itshape\hrule\smallskip#1\par\smallskip\hrule}}

\DeclareSymbolFont{ugmL}{OMX}{mdugm}{m}{n}
\SetSymbolFont{ugmL}{bold}{OMX}{mdugm}{b}{n}

\DeclareMathAccent{\wideparen}{\mathord}{ugmL}{"F3}


%\newcommand{\namelistlabel}[1]{\mbox{#1}\hfill}
%\newenvironment{namelist}[1]{%
%\begin{list}{}
%{
%     \let\makelabel\namelistlabel
%     \settowidth{\labelwidth}{#1}
%     \setlength{\leftmargin}{1.3\labelwidth}
%    }
%  }}%
%\end{list}}
\begin{document}
\bibliographystyle{plain}
\graphicspath{{./figs/}}
\baselineskip 13pt
\title{Mixed triangular spectral element method for elliptic problem}
\maketitle
\section{2-D problem}
%\subsection{Model problem and mixed element formulation}
%\begin{equation}
%\label{modelproblem}
%\left\{\begin{array}{ll}\displaystyle -\nabla\cdot(\beta\nabla
%u(\bs x))=f(\bs x)&\displaystyle \bs x\in\Omega\\
%\displaystyle u=g &\displaystyle on\quad \partial\Omega
%\end{array}\right.
%\end{equation}
%where $\beta(x,y)$ is a given function. For the sake of the definition of the mixed spectral elment scheme, we rewrite (\ref{modelproblem}) into a first order system
%\begin{eqnarray}
%\beta^{-1}{\bs q}-\nabla u=0 & \forall \bs x\in \Omega\nonumber\\
%-\nabla\cdot{\bs q}=f(\bs x) & \forall \bs x\in \Omega\label{firstordersystem}\\
%u=g & on\quad \partial\Omega\nonumber
%\end{eqnarray}
%The variational formulation is difined as: Find $(\bs q, u)\in (H^1(\Omega))^2\times H^1(\Omega)$ s.t.
%\begin{eqnarray}
%(\beta^{-1}{\bs q}, \bs v)_{\Omega}+(u, \nabla\cdot\bs v)_{\Omega}=&\langle g, \bs v\cdot\bs n\rangle_{\partial\Omega}\nonumber\\
%-(\nabla\cdot{\bs q},w)_{\Omega}=&(f, w)_{\Omega}\label{variationalform}
%\end{eqnarray}
%Define
%\begin{equation}
%{\bs W}_N=\{ v\in H^1(\Omega), v|_{\Omega_e}\in \mathcal{P}_N(\Omega_e)\},\quad \bs V_N=(\bs W_N)^2
%\end{equation}
%then the mixed spectral element formulation is: find ${\bs q}_N\in {\bs V}_N$, $u_N\in {\bs W}_N$, s.t.
%\begin{eqnarray}
%\left(\beta^{-1}{\bs q}_N, {\bs v}_N\right)+(u_N,\nabla\cdot{\bs v}_N)=\langle g,{\bs v}_h\cdot{\bs n}\rangle\nonumber\\
%-(\nabla\cdot{\bs q}_N, w_N)=(f,w_N)\label{localsolver}
%\end{eqnarray}
%for all ${\bs v}_N\in {\bs V}_N$ and $w_N\in {\bs W}_N$.
%
%\;\;
%
%\subsection{Assembling linear system}
%Denote by $\Phi_i\big|_{i=1}^{d_q}$, $\phi_i\big|_{i=1}^{d_u}$  are basis of spaces
%${\bs V}_N$, $\mathcal{W}_N$. Then ${\bs q}_N$, $u_N$ have expressions
%$${\bs q}_N=\sum q_j\Phi_j,\quad u_N=\sum u_j\varphi_j.$$
%where $\Phi_j$ has the form
%\begin{equation}
%\Phi_1^{(m)}=\begin{pmatrix}
%\varphi^{(m)}_1\\
%0
%\end{pmatrix},\cdots,
%\Phi_{N}^{(m)}=\begin{pmatrix}
%\varphi^{(m)}_{N}\\
%0
%\end{pmatrix},\quad
%\Phi_{N+1}^{(m)}=\begin{pmatrix}
%0\\
%\varphi^{(m)}_1
%\end{pmatrix},\quad \Phi_{2N}^{(m)}=\begin{pmatrix}
%0\\
%\varphi^{(m)}_{N}
%\end{pmatrix}.
%\end{equation}
%
%Substitute these expressions into (\ref{localsolver}) and set $v_N=\Phi_i$ and $w_N=\phi_i$ we have
%\begin{equation}
%\begin{split}
%\sum q_j\big(\beta^{-1}\Phi_j, \Phi_i\big)_{\Omega}+\sum\limits_{j=1}u_j\big(\varphi_j, \nabla\cdot\Phi_i\big)_{\Omega} =&\langle g, \Phi_i\cdot{\bs n}\rangle_{\partial\Omega} \nonumber\\
%-\sum q_j(\nabla\cdot\Phi_j,\varphi_i)_{{\Omega}}
%=&(f,\varphi_i)_{\Omega} \nonumber
%\end{split}
%\end{equation}
%Denoted by
%\begin{equation}
%\begin{array}{lll}
%\mathbb{M}=\big[(\beta^{-1}\varphi_i, \varphi_j)_{{\Omega}}\big],
%&\mathbb{C}_x=\big[(\partial_x\varphi_i, \varphi_j)_{{\Omega}}\big],
%&\mathbb{C}_y=\big[(\partial_y\varphi_i, \varphi_j)_{{\Omega}}\big]
%\end{array}
%\end{equation}
%then the formulation \eqref{localsolver} equivalent to the following linear systems
%\begin{equation}
%\label{locallinearsys}
%\begin{bmatrix}
%\mathbb{K}_{11} & \mathbb{K}_{12}  \\
%\mathbb{K}_{21} & \bs 0
%\end{bmatrix}
%\begin{bmatrix}
%\mathbb{Q}\\
%\mathbb{U}
%\end{bmatrix}=\mathbb{F}
%\end{equation}
%where
%\begin{equation}
%\mathbb K_{11}=\begin{pmatrix}
%\mathbb M & 0 \\
%0 & \mathbb M
%\end{pmatrix}\quad \mathbb K_{12}=\begin{pmatrix}
%\mathbb C_x \\
% \mathbb C_y
%\end{pmatrix}
%\quad \mathbb K_{21}=-\begin{pmatrix}
%\mathbb C_x & \mathbb C_y
%\end{pmatrix}
%\end{equation}
%

\subsection{Model problem and mixed element formulation}
\begin{equation}
\label{modelproblem1}
\left\{\begin{array}{ll}\displaystyle -\nabla\cdot(\beta\nabla
u(\bs x))  +\gamma u   =    f(\bs x)&\displaystyle \bs x\in\Omega\\
\displaystyle u=g &\displaystyle on\quad \partial\Omega
\end{array}\right.
\end{equation}
where $\beta(x,y)$ is a given function. For the sake of the definition of the mixed spectral elment scheme, we rewrite (\ref{modelproblem1})
into a first order system
\begin{eqnarray}
\beta^{-1}{\bs q}-\nabla u=0 & \forall \bs x\in \Omega\nonumber\\
-\nabla\cdot{\bs q}  +\gamma u   =f(\bs x) & \forall \bs x\in \Omega\label{firstordersystem1}\\
u=g & on\quad \partial\Omega\nonumber
\end{eqnarray}
The variational formulation is difined as: Find $(\bs q, u)\in (H^1(\Omega))^2\times H^1(\Omega)$ s.t.
\begin{eqnarray}
&(\beta^{-1}{\bs q}, \;\bs v)_{\Omega}  =(\nabla u, \;\bs v)_{\Omega}\nonumber\\
&({\bs q},\;\nabla w)_{\Omega}  + (\gamma u,\; w)_{\Omega}  = (f,\; w)_{\Omega}  \label{variationalform1}
%&({\bs q},\;\nabla w)_{\Omega}=(f,\; w)_{\Omega} + \langle g,\; w\rangle_{\partial \Omega} \label{variationalform}
%-(\nabla\cdot{\bs q},\;w)_{\Omega}=&(f,\; w)_{\Omega}\label{variationalform}
\end{eqnarray}

Define
\begin{equation}
{\bs W}_N=\{ v\in H^1(\Omega), v|_{\Omega_e}\in \mathcal{P}_N(\Omega_e)\},\quad \bs V_N=(\bs W_N)^2
\end{equation}
then the mixed spectral element formulation is: find ${\bs q}_N\in {\bs V}_N$, $u_N\in {\bs W}_N$, s.t.
\begin{eqnarray}
&(\beta^{-1}{\bs q_N}, \;\bs v_N)_{\Omega}  =(\nabla u_N, \;\bs v_N)_{\Omega}\nonumber\\
&({\bs q_N},\;\nabla w_N)_{\Omega} + (\gamma u_N,\; w_N)_{\Omega} =(f,\; w_N)_{\Omega} \label{localsolver}
\end{eqnarray}
for all ${\bs v}_N\in {\bs V}_N$ and $w_N\in {\bs W}_N$.

\subsection{Assembling linear system}
Denote by $\Phi_i\big|_{i=1}^{d_q}$, $\phi_i\big|_{i=1}^{d_u}$  are basis of spaces
${\bs V}_N$, $\mathcal{W}_N$. Then ${\bs q}_N$, $u_N$ have expressions
$${\bs q}_N=\sum q_j\Phi_j,\quad u_N=\sum u_j\varphi_j.$$
where $\Phi_j$ has the form
\begin{equation}
\Phi_1^{(m)}=\begin{pmatrix}
\varphi^{(m)}_1\\
0
\end{pmatrix},\cdots,
\Phi_{N}^{(m)}=\begin{pmatrix}
\varphi^{(m)}_{N}\\
0
\end{pmatrix},\quad
\Phi_{N+1}^{(m)}=\begin{pmatrix}
0\\
\varphi^{(m)}_1
\end{pmatrix},\quad \Phi_{2N}^{(m)}=\begin{pmatrix}
0\\
\varphi^{(m)}_{N}
\end{pmatrix}.
\end{equation}

Substitute these expressions into (\ref{localsolver}) and set $v_N=\Phi_i$ and $w_N=\phi_i$ we have
\begin{equation}
\begin{split}
\sum q_j\big(\beta^{-1}\Phi_j, \Phi_i\big)_{\Omega}
=&\sum u_j\big(\nabla\varphi_j,\Phi_i\big)_{\Omega} \nonumber\\
%=&\langle g, \Phi_i\cdot{\bs n}\rangle_{\partial\Omega}
\sum q_j(\Phi_j,\nabla\varphi_i)_{{\Omega}} + \sum u_j\big( \gamma\varphi_j,   \varphi_i\big)_{\Omega}
=&(f,\varphi_i)_{\Omega} \nonumber
\end{split}
\end{equation}
Denoted by
\begin{equation*}
\begin{array}{lll}
\mathbb{M}=\big[(\beta^{-1}\varphi_i, \varphi_j)_{{\Omega}}\big],
&\mathbb{N}=\big[(\gamma\varphi_i, \varphi_j)_{{\Omega}}\big],
&{\bf F}=\big[ (f,\varphi_i)_{\Omega} \big]
\end{array}
\end{equation*}
\begin{equation*}
 \mathbb{C}_x=\big[(\varphi_i, \partial_x\varphi_j)_{{\Omega}}\big], \quad  \mathbb{C}_y=\big[(\varphi_i, \partial_y\varphi_j)_{{\Omega}}\big]
\end{equation*}
\begin{equation*}
  {\bf Q}_x = [q_1,q_2,\cdots,q_N]^T, \quad  {\bf Q}_y = [q_{N+1},q_{N+2},\cdots,q_{2N}]^T, \quad  {\bf U} = [u_1, u_2,\cdots, u_N]^T
\end{equation*}

then the formulation \eqref{localsolver} equivalent to the following linear systems
\begin{equation}
\label{locallinearsys}
\begin{split}
\mathbb{M}  {\bf Q}_x &=  \mathbb{C}_x  {\bf U}\\
\mathbb{M}  {\bf Q}_y &=  \mathbb{C}_y  {\bf U}\\
 \mathbb{C}_x^T  {\bf Q}_x + \mathbb{C}_y^T  {\bf Q}_y  + \mathbb{N} &= {\bf F}
\end{split}
\end{equation}
or
\begin{equation*}
\Bigg(
  \begin{matrix}
  -\mathbb{M}     &  {\bs 0}        & \mathbb{C}_x   \\
   {\bs 0}&    -\mathbb{M}      &\mathbb{C}_y\\
     \mathbb{C}_x^T & \mathbb{C}_y^T  & {\mathbb{N}}\\
  \end{matrix}
  \Bigg)
\Bigg(
  \begin{matrix}
  {\bf Q}_x \\
  {\bf Q}_y \\
  {\bf U}
  \end{matrix}
  \Bigg)
  {=}
\Bigg(
  \begin{matrix}
  {\bf 0} \\
  {\bf 0} \\
  {\bf F}
  \end{matrix}
  \Bigg)
\end{equation*}

or the following form
\begin{equation*}
  \Big( \mathbb{C}_x^T  \mathbb{M}^{-1}  \mathbb{C}_x + \mathbb{C}_y^T   \mathbb{M}^{-1} \mathbb{C}_y  +\mathbb{N}  \Big)  {\bf U}= {\bf F}
\end{equation*}



\section{Numerical Test:}
Let $\gamma(x)=1.0, \;\beta = e^{x+y},\; \Omega = [0,1]^2$, and choose exact smooth solution
\begin{equation}
U(x)=\cos(\pi r^2),\quad r=\sqrt{(x^2+y^2)}
\end{equation}
Then
\begin{equation}
f = \Big( 2\pi(x+y+2)\sin(\pi r^2)+4\pi^2r^2\cos(\pi r^2) \Big)e^{x+y} + \cos(\pi r^2)
\end{equation}
and the other one exact nonsmooth solution
\begin{equation}
U(x)=r^5 ,\quad r=\sqrt{(x+y)}
\end{equation}
Then
\begin{equation}
f = -\Big(  5r^3 + \frac{15}{2}r  \Big)e^{x+y} + r^5
\end{equation}





\end{document}
